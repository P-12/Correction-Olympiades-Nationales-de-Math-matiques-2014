\documentclass[10pt,a4paper]{article}
\usepackage[utf8]{inputenc}  
\usepackage[T1]{fontenc}       
\usepackage[francais]{babel}
\usepackage{fullpage}
%\usepackage{euler}
\usepackage{amsmath}
%\usepackage{framed}
\usepackage{float}
\usepackage{amsfonts}
\usepackage{amssymb}
\usepackage{pifont}
\usepackage{mathrsfs}
\usepackage[squaren,Gray]{SIunits}
\usepackage{graphicx}
\usepackage{pstricks-add}
\usepackage[a4paper]{geometry}
\geometry{hscale=0.86,vscale=0.87,centering}

\pagestyle{empty}

%\newcommand{\Titre}[4]{\noindent \textsc{#1}} \hfill \textbf{\textsc{#2}}\\ #3 \hfill \emph{#4}\\  \hrule\vspace{\baselineskip}}

\newcommand{\Titre}[3]{\begin{center} {\LARGE\textbf{\textsc{#1}}}\\ #2 \hfill \emph{#3} \\  \hrule\vspace{\baselineskip}\end{center}}

\begin{document}


\Titre{Correction des Olympiades Nationales de Mathématiques 2014}{Juliette Buet}{mars 2014}
\thispagestyle{plain}
\pagestyle{plain}


%\title{DM de Mathématiques}
%\date{19 novembre 2013}
%\author{Asp. J.Buet}
%\maketitle{}
\section{\textsc{Figures équilibrées}}
\begin{enumerate}
\item  
\begin{enumerate}
\item Voici une figure contenant 7 points marqués et 5 droites :

\psset{xunit=1.0cm,yunit=1.0cm}
\begin{pspicture*}(-2,-2)(7.1,5)
\psset{xunit=1.0cm,yunit=1.0cm,algebraic=true,dotstyle=o,dotsize=3pt 0,linewidth=0.8pt,arrowsize=3pt 2,arrowinset=0.25}
\psplot{-2}{7.1}{(--15-0*x)/5}
\psline(2,-2)(2,5)
\psplot{-2}{7.1}{(-1--2*x)/3}
\psplot{-2}{7.1}{(--6-2*x)/2}
\psplot{-2}{7.1}{(--11-4*x)/-3}
\begin{scriptsize}
\psdots[dotsize=5pt 0,dotstyle=*,linecolor=blue](0,3)
\psdots[dotsize=5pt 0,dotstyle=*,linecolor=blue](5,3)
\psdots[dotsize=5pt 0,dotstyle=*,linecolor=blue](2,3)
\psdots[dotsize=5pt 0,dotstyle=*,linecolor=blue](2,1)
\psdots[dotsize=5pt 0,dotstyle=*,linecolor=blue](2,-1)
\psdots[dotsize=5pt 0,dotstyle=*,linecolor=blue](2.86,0.14)
\psdots[dotsize=5pt 0,dotstyle=*,linecolor=blue](6.68,4.12)
\end{scriptsize}
\end{pspicture*}
\item Voici une figure contenant 9 points marqués et 8 droites :

\psset{xunit=1.0cm,yunit=1.0cm}
\begin{pspicture*}(-4.4,-1)(6,6)
\psset{xunit=1.0cm,yunit=1.0cm,algebraic=true,dotstyle=o,dotsize=3pt 0,linewidth=0.8pt,arrowsize=3pt 2,arrowinset=0.25}
\psplot{-4.4}{6}{(--12-0*x)/4}
\psline(1,-1)(1,6)
\psplot{-4.4}{6}{(--4-2*x)/2}
\psplot{-4.4}{6}{(-8-2*x)/-2}
\psplot{-4.4}{6}{(--12-2*x)/2}
\psplot{-4.4}{6}{(-0--2*x)/2}
\psplot{-4.4}{6}{(-10-2*x)/-4}
\psplot{-4.4}{6}{(--12--3*x)/7}
\begin{scriptsize}
\psdots[dotsize=5pt 0,dotstyle=*,linecolor=blue](-1,3)
\psdots[dotsize=5pt 0,dotstyle=*,linecolor=blue](3,3)
\psdots[dotsize=5pt 0,dotstyle=*,linecolor=blue](1,5)
\psdots[dotsize=5pt 0,dotstyle=*,linecolor=blue](1,1)
\psdots[dotsize=5pt 0,dotstyle=*,linecolor=blue](5,5)
\psdots[dotsize=5pt 0,dotstyle=*,linecolor=blue](1,3)
\psdots[dotsize=5pt 0,dotstyle=*,linecolor=blue](-4,0)
\psdots[dotsize=5pt 0,dotstyle=*,linecolor=blue](2.33,3.67)
\psdots[dotsize=5pt 0,dotstyle=*,linecolor=blue](0.2,1.8)
\end{scriptsize}
\end{pspicture*}
\end{enumerate}
\item Voici une numérotation qui n'est pas magique :

\begin{minipage}{0.4\textwidth}
\psset{xunit=1.0cm,yunit=1.0cm}
\begin{pspicture*}(-1,0)(4,4)
\psset{xunit=1.0cm,yunit=1.0cm,algebraic=true,dotstyle=o,dotsize=3pt 0,linewidth=0.8pt,arrowsize=3pt 2,arrowinset=0.25}
\psplot{-1}{3}{(--4--1*x)/2}
\psplot{-1}{3}{(--4-0*x)/2}
\begin{scriptsize}
\psdots[dotstyle=*,linecolor=blue](0,2)
\rput[bl](0.00,2.25){\large\black{$1$}}
\psdots[dotstyle=*,linecolor=blue](2,3)
\rput[bl](2.0,3.25){\large\black{$2$}}
\psdots[dotstyle=*,linecolor=blue](2,2)
\rput[bl](2.08,2.12){\large\black{$4$}}
\psdots[dotstyle=*,linecolor=blue](2.98,3.49)
\rput[bl](3.0,3.7){\large\black{$3$}}
\psdots[dotstyle=*,linecolor=blue](3,2)
\rput[bl](3.08,2.12){\large\black{$5$}}
\end{scriptsize}
\end{pspicture*}
\end{minipage}
\begin{minipage}{0.4\textwidth}
En effet\,: $1+2+3=6\neq 1+4+5 =10$.
\end{minipage}

Voici une numérotation magique de cette figure avec $K=10$ :

\begin{minipage}{0.4\textwidth}
\psset{xunit=1.0cm,yunit=1.0cm}
\begin{pspicture*}(-1,0)(4,4)
\psset{xunit=1.0cm,yunit=1.0cm,algebraic=true,dotstyle=o,dotsize=3pt 0,linewidth=0.8pt,arrowsize=3pt 2,arrowinset=0.25}
\psplot{-1}{3}{(--4--1*x)/2}
\psplot{-1}{3}{(--4-0*x)/2}
\begin{scriptsize}
\psdots[dotstyle=*,linecolor=blue](0,2)
\rput[bl](0.0,2.25){\large\black{$5$}}
\psdots[dotstyle=*,linecolor=blue](2,3)
\rput[bl](2.0,3.25){\large\black{$2$}}
\psdots[dotstyle=*,linecolor=blue](2,2)
\rput[bl](2.08,2.12){\large\black{$4$}}
\psdots[dotstyle=*,linecolor=blue](2.98,3.49)
\rput[bl](3.06,3.62){\large\black{$3$}}
\psdots[dotstyle=*,linecolor=blue](3,2)
\rput[bl](3.08,2.12){\large\black{$1$}}
\end{scriptsize}
\end{pspicture*}
\end{minipage}
\begin{minipage}{0.4\textwidth}
En effet\,: $5+4+1=10=5+2+3$.
\end{minipage}
\item 
\begin{enumerate}
\item On suppose que la figure est magique. On obtient donc le système suivant\,: 
$$\begin{cases}
b+d+e&=K \\
d+c+f&=K \\
b+a+f&=K\\
e+a+c&=K
\end{cases}$$

On somme ces quatre équations et on obtient\,:
$$2a+2b+2c+2d+2e+2f=4K$$
$$2(a+b+c+d+e+f)=4K$$
$$2\times(1+2+3+4+5+6)=4K$$
$$\boxed{4K=42}$$
\item On ne peut pas trouver une numérotation magique de cette figure car $42$ n'est pas divisible par $4$, donc on ne peut pas trouver un $K$ entier tel que
$4K=42$.
\end{enumerate}
\item 
\begin{enumerate}
\item Les entiers $\{a,c,e\}$ correspondent à trois entiers distincts de l'ensemble $\{1,2,3,4,5,6\}$. Le triplet de plus petite somme est $\{1,2,3\}$ et 
celui de plus grande somme est $\{4,5,6\}$.

Donc au minimum $a+c+e=1+2+3=6$ et au maximum $a+c+e=4+5+6=15$.
$$\boxed{6\leqslant a+c+e \leqslant 15}$$
\item On suppose que la numérotation de cette figure est magique. On obtient le système suivant :
$$\begin{cases}
a+b+b&=K\\
c+d+e&=K\\
e+f+a&=K
\end{cases}$$
On somme ces trois équations et on obtient :
$$2a+2e+2c+b+d+f=3K$$
$$a+c+e+(a+b+c+d+e+f)=3K$$
$$a+c+e=3K-(1+2+3+4+5+6)$$
$$a+c+e=3K-21$$
$$\boxed{a+c+e=3(K-7)}$$
\item D'après les deux questions précédentes, on a\, :
$$6\leqslant 3(K-7)\leqslant 15$$
$$2\leqslant K-7\leqslant 5$$
$$9\leqslant K\leqslant 12$$
Nous avons raisonné par implication, il faut maintenant montrer que l'on peut trouver des figures magiques pour chacune des quatre constantes possibles.


\begin{figure}[H]
\begin{minipage}{0.22\textwidth}
\psset{xunit=1.0cm,yunit=1.0cm}
\begin{pspicture*}(-1,0)(2,4.3)
\psset{xunit=0.5cm,yunit=0.8cm,algebraic=true,dotstyle=o,dotsize=5pt 0,linewidth=0.8pt,arrowsize=3pt 2,arrowinset=0.25}
\psplot{-2}{4}{(-5-3*x)/-2}
\psplot{-2}{4}{(--4-0*x)/4}
\psplot{-2}{4}{(-11--3*x)/-2}
\begin{scriptsize}
\psdots[dotstyle=*,linecolor=blue](1,4)
\rput[bl](0.8,4.25){\large\black{$1$}}
\psdots[dotstyle=*,linecolor=blue](-1,1)
\rput[bl](-0.7,1.2){\large\black{$3$}}
\psdots[dotstyle=*,linecolor=blue](3,1)
\rput[bl](3.1,1.26){\large\black{$2$}}
\psdots[dotstyle=*,linecolor=blue](-0.02,2.46)
\rput[bl](0.25,2.6){\large\black{$5$}}
\psdots[dotstyle=*,linecolor=blue](2.04,2.44)
\rput[bl](2.11,2.6){\large\black{$6$}}
\psdots[dotstyle=*,linecolor=blue](1,1)
\rput[bl](1.1,1.26){\large\black{$4$}}
\end{scriptsize}
\end{pspicture*}
\caption{K=9}
\end{minipage}
\hfill
\begin{minipage}{0.22\textwidth}
\psset{xunit=1.0cm,yunit=1.0cm}
\begin{pspicture*}(-1,0)(2,4.3)
\psset{xunit=0.5cm,yunit=0.8cm,algebraic=true,dotstyle=o,dotsize=5pt 0,linewidth=0.8pt,arrowsize=3pt 2,arrowinset=0.25}
\psplot{-2}{4}{(-5-3*x)/-2}
\psplot{-2}{4}{(--4-0*x)/4}
\psplot{-2}{4}{(-11--3*x)/-2}
\begin{scriptsize}
\psdots[dotstyle=*,linecolor=blue](1,4)
\rput[bl](0.8,4.25){\large\black{$1$}}
\psdots[dotstyle=*,linecolor=blue](-1,1)
\rput[bl](-0.7,1.2){\large\black{$3$}}
\psdots[dotstyle=*,linecolor=blue](3,1)
\rput[bl](3.1,1.26){\large\black{$5$}}
\psdots[dotstyle=*,linecolor=blue](-0.02,2.46)
\rput[bl](0.25,2.6){\large\black{$6$}}
\psdots[dotstyle=*,linecolor=blue](2.04,2.44)
\rput[bl](2.11,2.6){\large\black{$4$}}
\psdots[dotstyle=*,linecolor=blue](1,1)
\rput[bl](1.1,1.26){\large\black{$2$}}
\end{scriptsize}
\end{pspicture*}
\caption{K=10}
\end{minipage}
\hfill
\begin{minipage}{0.22\textwidth}
\psset{xunit=1.0cm,yunit=1.0cm}
\begin{pspicture*}(-1,0)(2,4.3)
\psset{xunit=0.5cm,yunit=0.8cm,algebraic=true,dotstyle=o,dotsize=5pt 0,linewidth=0.8pt,arrowsize=3pt 2,arrowinset=0.25}
\psplot{-2}{4}{(-5-3*x)/-2}
\psplot{-2}{4}{(--4-0*x)/4}
\psplot{-2}{4}{(-11--3*x)/-2}
\begin{scriptsize}
\psdots[dotstyle=*,linecolor=blue](1,4)
\rput[bl](0.8,4.25){\large\black{$2$}}
\psdots[dotstyle=*,linecolor=blue](-1,1)
\rput[bl](-0.7,1.2){\large\black{$4$}}
\psdots[dotstyle=*,linecolor=blue](3,1)
\rput[bl](3.1,1.26){\large\black{$6$}}
\psdots[dotstyle=*,linecolor=blue](-0.02,2.46)
\rput[bl](0.25,2.6){\large\black{$5$}}
\psdots[dotstyle=*,linecolor=blue](2.04,2.44)
\rput[bl](2.11,2.6){\large\black{$3$}}
\psdots[dotstyle=*,linecolor=blue](1,1)
\rput[bl](1.1,1.26){\large\black{$1$}}
\end{scriptsize}
\end{pspicture*}
\caption{K=11}
\end{minipage}
\hfill
\begin{minipage}{0.22\textwidth}
\psset{xunit=1.0cm,yunit=1.0cm}
\begin{pspicture*}(-1,0)(2,4.3)
\psset{xunit=0.5cm,yunit=0.8cm,algebraic=true,dotstyle=o,dotsize=5pt 0,linewidth=0.8pt,arrowsize=3pt 2,arrowinset=0.25}
\psplot{-2}{4}{(-5-3*x)/-2}
\psplot{-2}{4}{(--4-0*x)/4}
\psplot{-2}{4}{(-11--3*x)/-2}
\begin{scriptsize}
\psdots[dotstyle=*,linecolor=blue](1,4)
\rput[bl](0.8,4.25){\large\black{$6$}}
\psdots[dotstyle=*,linecolor=blue](-1,1)
\rput[bl](-0.7,1.2){\large\black{$4$}}
\psdots[dotstyle=*,linecolor=blue](3,1)
\rput[bl](3.1,1.26){\large\black{$5$}}
\psdots[dotstyle=*,linecolor=blue](-0.02,2.46)
\rput[bl](0.25,2.6){\large\black{$2$}}
\psdots[dotstyle=*,linecolor=blue](2.04,2.44)
\rput[bl](2.11,2.6){\large\black{$1$}}
\psdots[dotstyle=*,linecolor=blue](1,1)
\rput[bl](1.1,1.26){\large\black{$3$}}
\end{scriptsize}
\end{pspicture*}
\caption{K=12}
\end{minipage}
\end{figure}


Je trouve bien une figure magique pour chacune de ces quatre constantes et on a montré que s'il existait des constantes magiques, alors elles étaient 
comprises entre $9$ et $12$. Finalement, on a bien quatre constantes magiques qui sont $\boxed{\{9,10,11,12\}}$.
\end{enumerate}
\item Montrons que cette figure n'admet pas de numérotation magique.
On remarque d'abord que\,: 
\begin{itemize}
\item chaque point appartient à au moins $3$ droites
\item les trois points centraux sont reliés à tous les autres points
\end{itemize}
Supposons que la figure est magique de constante $K$.

La plus grande somme que l'on peut faire sur la droite contenant la valeur $1$ est $1+8+9=18$\,; la plus petite somme que l'on peut faire sur la droite 
contenant la valeur $9$ est $9+1+2=12$. Donc  on a $12\leqslant K\leqslant 18$.

Appelons $a,b,c$ les trois valeurs des points centraux. $1$ ne peut pas être une de ces valeurs. En effet, si $1$ était une de ces valeurs, la constante 
magique serait nécessairement $12$ ($1$ est relié à $9$ et à $2$, si $9$ et $2$ ne sont pas sur la même droite on a au mieux $9+1+3=13>8+2+1=11$, donc la 
figure ne peut être magique que si $9$ et $2$ sont sur la même droite). $9$ est aussi relié à deux autres droites et il est impossible de faire $12$ avec 
les chiffres restants. Donc si la figure est magique, $1$ ne peut pas être la valeur d'un des trois points centraux.

Avec un raisonnement analogue, on montre que $9$ ne peut pas être une des trois valeurs centrales non plus.

Notons $d,e,f$ les trois chiffres sur les droites passant par $1$ et par $a,b$ ou $c$, alors\,:

\begin{minipage}{0.3\textwidth}
$$\begin{cases}
a+1+d&=K \\
b+1+e&=K\\
c+1+f&=K
\end{cases}$$
\end{minipage}
\hfill
\begin{minipage}{0.5\textwidth}
\psset{xunit=1.0cm,yunit=1.0cm}
\begin{pspicture*}(-3.5,-2)(5.5,3.5)
\psset{xunit=0.5cm,yunit=0.5cm,algebraic=true,dotstyle=o,dotsize=5pt 0,linewidth=0.8pt,arrowsize=3pt 2,arrowinset=0.25}
\psplot{-7}{11}{(--16-0*x)/8}
\psline(2,-7)(2,7)
\psplot{-7}{11}{(-16-4*x)/-4}
\psplot{-7}{11}{(--32-4*x)/4}
\psplot{-7}{11}{(-0--4*x)/-4}
\psplot{-7}{11}{(-16--4*x)/4}
\psplot{-7}{11}{(--16-0*x)/-8}
\psplot{-7}{11}{(-48-0*x)/-8}
\psplot{-7}{11}{(--48-8*x)/16}
\psplot{-7}{11}{(-8-4*x)/-8}
\begin{scriptsize}
\psdots[dotstyle=*,linecolor=blue](-2,2)
\rput[bl](-2.1,2.25){\large\black{$f$}}
\psdots[dotstyle=*,linecolor=blue](6,2)
\rput[bl](6.17,2.25){\large\black{$$}}
\psdots[dotstyle=*,linecolor=blue](2,2)
\rput[bl](2.17,2.25){\large\black{$b$}}
\psdots[dotstyle=*,linecolor=blue](2,6)
\rput[bl](2.5,6.2){\large\black{$a$}}
\psdots[dotstyle=*,linecolor=blue](2,-2)
\rput[bl](2.45,-1.85){\large\black{$c$}}
\psdots[dotstyle=*,linecolor=blue](-6,-2)
\rput[bl](-5.8,-1.85){\large\black{$$}}
\psdots[dotstyle=*,linecolor=blue](-6,6)
\rput[bl](-5.8,6.2){\large\black{$1$}}
\psdots[dotstyle=*,linecolor=blue](10,6)
\rput[bl](9.8,6.2){\large\black{$d$}}
\psdots[dotstyle=*,linecolor=darkgray](10,-2)
\rput[bl](10.18,-1.71){\large\black{$e$}}
\end{scriptsize}
\end{pspicture*}
\end{minipage}

En sommant les trois on obtient :$a+b+c+3+d+e+f=3K$. Or $a+b+c=K$, donc $d+e+f+3=2K$. On a également $d+e+f\leqslant 9+8+7=24$. Donc on a forcément 
$2K\leqslant 27$ or $K$ est entier donc $K\leqslant 13$.

De la même manière, en notant $d',e'$ et $f'$ les trois entiers situés sur les droites contenant $9$ et un des entiers $a,b,c$, on a :
$$\begin{cases}
a+9+d'&=K\\
b+9+e'&=K\\
c+9+f'&=K
\end{cases}$$
En sommant les trois équations on obtient $a+b+c+27+d'+e'+f'=3K$ soit $d'+e'+f'+27=2K$. Or $d'+e'+f'\geqslant 1+2+3=6$, donc on a forcément $2K\geqslant 33$
soit $K\geqslant 17$ car $K$ est un entier.

On a donc nécessairement $K\leqslant 13$ et $K\geqslant 17$. Ce qui est absurde.

Donc il n'existe pas de numérotation magique de la figure.
\end{enumerate}

\pagebreak
\section{\textsc{Le plus court possible}}
\subsection{\textsc{Partie A}}
\begin{enumerate}
\item Calculons la longueur de chacune des trois routes proposées.
\begin{itemize}
\item
FIG-1 : $L=AB+BC+CD=\unit{300}{\kilo\meter}$
\item FIG-2 : $L=AC+BD=2\sqrt{2}\times 100\simeq \unit{283}{\kilo\meter}$
\item FIG-3 : $L=\pi\times BE+AF+DG=\pi BE+AE-FE+DE-GE=\pi BE+2\times(\sqrt{BE^2+AB^2})-2\times BE = 50\pi+100\sqrt{5}-100\simeq\unit{281}{\kilo\meter}$
\end{itemize}
L'assistant numéro 3 propose le réseau routier le plus court.
\item Calculons $AE$. La médiane issue de $E$ est aussi une hauteur du triangle $AED$. Elle a la même longueur que la médiane issue de $F$ soit 
$\frac{100-20}{2}=\unit{40}{\kilo\meter}$.
$$AE=\sqrt{40^2+50^2}\simeq \unit{64}{\kilo\meter}$$
\underline{Le chemin a donc une longueur\,: $4\times\sqrt{4100}+20\simeq\unit{276}{\kilo\meter}$. Il est donc plus court.}
\end{enumerate}
\subsection{\textsc{Partie B}}
\begin{enumerate}
\item Comme rappelé dans l'énoncé, le chemin le plus court entre deux points est la ligne droite. Donc la courbe reliant $A$ à $E_0$ sur la figure $5$ est
 plus longue que le segment $[AE_0]$ sur la figure $6$. De même pour les courbes reliant $D$ à $E_0$, $B$ à $F_0$ et $C$ à $F_0$.

La courbe en trait plein reliant $E_0$ à $F_0$ sur la figure $5$ est plus longue que le segment reliant $E_0$ à $F_0$. De même pour la courbe en pointillé. A
fortiori, le réseau routier entre $E_0$ et $F_00$ est plus long sur la figure $5$ que sur la figure $6$.

Finalement, chaque portion de courbe est plus longue sur la figure $5$ que sur la figure $6$, donc le réseau routier total est plus long sur la figure $5$ que
sur la figure $6$.
\vspace{0.3cm}

\begin{minipage}{0.65\textwidth}
\item
\begin{enumerate}
\item 
On trace $D'$ le symétrique du point $D$ par rapport à $\Delta_E$. On a alors $E_0D'=E_0D$. On trace la droite $(AD')$. Elle coupe $\Delta_E$ en un point $M$.

Minimiser $AE_0+E_0D$ revient à minimiser $AE_0+E_0D'$. Comme rappelé au début de l'énoncé $AE_0+E_0D'\geqslant AD'$ et il y a égalité si $E_0$ appartient au
segment $[AD']$, c'est-à-dire si $E_0$ est au point $M$.

Finalement, le point $E$ correspond au point $M$ de la figure. C'est-à-dire le point qui est sur la médiatrice de $[AD]$ et sur $\Delta_E$. C'est aussi le
 point tel que le triangle $AED$ soit isocèle.
\item La droite $(EF)$ est perpendiculaire aux deux droites $\Delta_E$ et $\Delta_F$. Donc la longueur $EF$ correspond à la distance entre ces deux droites, 
c'est-à-dire la plus petite distance reliant deux points de ces droites. Donc $EF\leqslant E_0F_0$.
\item Les points $E$ et $F$ permettent de minimiser à la fois $AE+ED$, $BF+FC$ et $EF$. Donc chacune de ces trois portions du réseau est plus courte sur la
figure $8$ que sur la figure $7$. Donc le réseau cherché est nécessairement tel que représenté sur la figure $8$.
\end{enumerate}
\end{minipage}
\hspace{0.5cm}
\begin{minipage}{0.3\textwidth}
\psset{xunit=1.0cm,yunit=1.0cm}
\begin{pspicture*}(-2.5,1)(2.5,7)
\psset{xunit=1.0cm,yunit=1.0cm,algebraic=true,dotstyle=o,dotsize=3pt 0,linewidth=0.8pt,arrowsize=3pt 2,arrowinset=0.25}
\psline(-2,2)(2,2)
\psplot{-2.5}{3}{(--28-0*x)/7}
\psplot{-2.5}{3}{(--16--4*x)/4}
\begin{scriptsize}
\psdots[dotstyle=*,linecolor=blue](-2,2)
\rput[bl](-2.2,2.11){\large\black{$A$}}
\psdots[dotstyle=*,linecolor=blue](2,2)
\rput[bl](2.07,2.11){\large\black{$D$}}
\psdots[dotstyle=*,linecolor=blue](-1,4)
\rput[bl](-1.3,4.1){\large\black{$E_0$}}
\psdots[dotstyle=*,linecolor=blue](6,4)
\rput[bl](6.07,4.1){\large\black{$D$}}
\rput[bl](-2.4,3.6){\large{$\Delta_E$}}
\psdots[dotstyle=*,linecolor=blue](2,6)
\rput[bl](1.8,6.11){\large\black{$D'$}}
\psdots[dotstyle=*,linecolor=blue](0,4)
\rput[bl](-0.3,4.1){\large\black{$M$}}
\end{scriptsize}
\end{pspicture*}
\end{minipage}
\item
\begin{enumerate}
\item La médiatrice de $[AB]$ est un axe de symétrie de la distribution de départ donc le réseau routier idéal doit aussi être symétrique par rapport à cet 
axe.
\item Appelons $x$ la longueur $EF$. Calculons la longueur du réseau routier en fonction de $x$.

$$L(x)=x+4\sqrt{50^2+(\frac{100-x}{2})^2}$$

Calculons la dérivée de $L$ et cherchons quand elle s'annule\,:
$$L'(x)=1+4\frac{\frac{(x-100)}{4}}{\sqrt{50^2+(\frac{100-x}{2})^2}}$$
$$L'(x)=1+\frac{x-100}{\sqrt{50^2+(\frac{100-x}{2})^2}}$$
$$L'(x)=1+2\frac{x-100}{\sqrt{100^2+(100-x)^2}}$$
Pour que $L'$ s'annule, il faut que $2\frac{x-100}{\sqrt{100^2+(100-x)^2}}=-1$.
$$2(x-100)=-\sqrt{100^2+(100-x)^2}$$
$$4(100-x)^2=100^2+(100-x)^2$$
$$3x^2-600x+2\times 100^2=0$$
On résout cette équation du second degré et on trouve deux solutions, dont une où $x>100$. On ne garde que celle où $0\leqslant x \leqslant 100$ qui vaut 
$100(1-\frac{\sqrt{3}}{3})$.

Cette solution correspond bien à un minimum de $L$ car par exemple pour $x=10$ on a $L'<0$ et pour $x=80$ on a $L'>0$.

$$\boxed{EF=100(1-\frac{\sqrt{3}}{3})}$$ 
\item Calculons l'angle entre $(DE)$ et (EF), c'est la moitié de l'angle $\widehat{DEA}$.
$$\tan(\overrightarrow{DE},\overrightarrow{EF})=\frac{50}{(100-EF)/2}=\frac{50}{50(1-1+\frac{\sqrt{3}}{3})}=\frac{1}{\frac{\sqrt{3}}{3}}=\sqrt{3}$$
$$(\overrightarrow{DE},\overrightarrow{EF})=\arctan(\frac{\sqrt{3}/2}{1/2})=\frac{\pi}{3}$$
$$\boxed{\widehat{DEA}=\frac{2\pi}{3}}$$
\end{enumerate}
\end{enumerate}
\end{document}
